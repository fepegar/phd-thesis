\chapter{Abstract}

% The signed declaration should be followed by an abstract consisting of no more than 300 words.

Epilepsy is the most common neurological disorder, affecting around 1\% of the population.
One third of epilepsies are drug-resistant.
If the epileptogenic zone can be localized precisely, curative resective surgery may be performed.
However, only 40 to 70\% of patients remain seizure-free after the surgery.
% Clinical pathway
Presurgical evaluation is a complex, multimodal process that requires subjective clinical decisions, often relying mostly on the multidisciplinary team's experience.
Thus, the clinical pathway could benefit from data-driven methods for clinical decision support.

% Machines/ML could help
In the last decade, deep learning has seen great advancements thanks to the improvement of \acp{GPU}, the development of new algorithms and the large amounts of data available for training.
% We don't have much data
Using deep learning in clinical settings is challenging as large datasets are rare due to privacy concerns and expensive annotation processes.

% We worked on this and that
This thesis introduces computational methods that pave the way towards a more data-driven clinical pathway for the treatment of epilepsy, overcoming the relatively small datasets available.
We used transfer learning from human action recognition datasets to characterize epileptic seizures from video-telemetry data.
We developed a software framework to predict the location of the epileptogenic zone given seizure semiologies, based on retrospective information from the literature.
We trained models using self-supervised and semi-supervised learning to perform quantitative analysis of resective surgery by segmenting resection cavities on brain \acp{MRI}.
Finally, we shared datasets and software tools that will accelerate research in medical image computing, particularly in the field of epilepsy.
