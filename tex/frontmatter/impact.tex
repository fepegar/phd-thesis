\chapter{Impact Statement}

The statement should describe, in no more than 500 words, how the expertise, knowledge, analysis,
discovery or insight presented in your thesis could be put to a beneficial use. Consider benefits both
inside and outside academia and the ways in which these benefits could be brought about.

The benefits inside academia could be to the discipline and future scholarship, research methods
or methodology, the curriculum; they might be within your research area and potentially within other
research areas.

The benefits outside academia could occur to commercial activity, social enterprise, professional
practice, clinical use, public health, public policy design, public service delivery, laws, public
discourse, culture, the quality of the environment or quality of life.

The impact could occur locally, regionally, nationally or internationally, to individuals, communities
or organisations and could be immediate or occur incrementally, in the context of a broader field of
research, over many years, decades or longer.

Impact could be brought about through disseminating outputs (either in scholarly journals or
elsewhere such as specialist or mainstream media), education, public engagement, translational
research, commercial and social enterprise activity, engaging with public policy makers and public
service delivery practitioners, influencing ministers, collaborating with academics and non-academics
etc.

Further information including a searchable list of hundreds of examples of UCL impact outside
of academia please see Research Impact\fnurl{https://www.ucl.ac.uk/impact/} website. For thousands more examples, please see
REF2014\fnurl{https://results.ref.ac.uk/(S(trbo3sw0ose3zbhpqrknx4rc))/Results/SelectUoa} website.


TorchIO is used as a research tool at numerous hospitals and leading academic institutions in various countries including the UK, the US, Switzerland, Germany, France, Russia, China, Korea, Canada, the Netherlands, Sweden, Spain, Italy, Australia, India, Belgium, Singapore, Iran, Pakistan and Austria.
