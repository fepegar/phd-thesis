\chapter{Impact statement}

The algorithms and software developed in this thesis have potential benefits both outside and inside academia.
The publications and tools that I have shared and present here are already accelerating research and facilitating clinical practice.

Inside the academic environment, our work is already being used and built upon by multiple institutions.
We curated the EPISURG dataset, comprising 699 \acp{MRI} from 430 patients who underwent epilepsy surgery at the \ac{NHNN} between 1990 and 2018, including 200 manual annotations from three different human raters.
To the best of our knowledge, EPISURG is the first open annotated database of post-resection \ac{MRI} for epilepsy patients.
% EPISURG and the dataset of seizure videos features have been made open-access and can be freely downloaded from the UCL Research Data Repository \cite{perez-garcia_episurg_2020,perez-garcia_data_2021}.
EPISURG has been made open-access and can be freely downloaded from the UCL Research Data Repository \cite{perez-garcia_episurg_2020}.
As we proposed in \cite{perez-garcia_self-supervised_2021} (a collaboration between the \ac{NHNN} and hospitals in Milan, Marseille and Paris), our method to simulate brain lesions is being used in research for brain tumor segmentation \cite{zhang_self-supervised_2021}, which demonstrates its generalizability and potential to solve related problems.

TorchIO, our software library for medical image processing,
% has been used as a research tool
has been cited over \torchiocitations by researchers
at numerous hospitals and leading academic institutions in various countries including
the United Kingdom, the United States, Canada, Australia, Japan, China, Korea, Singapore, India, Iran, Pakistan, Saudi Arabia, Egypt, Kazakhstan, Russia, Switzerland, Germany, Austria, France, Sweden, Spain, Italy, Belgium and the Netherlands%
% accumulating over \torchiocitations citations%
\fnurl{https://scholar.google.co.uk/scholar?oi=bibs\&cites=11818021599290863762}.
The package is downloaded from the \ac{PyPI} over \torchiomonthdownloads times per month%
\fnurl{https://pypistats.org/packages/torchio}%
and the GitHub repository has over \torchiostars stars%
\fnurl{https://github.com/fepegar/torchio}.
I was invited to present TorchIO at
the UCL \ac{CMIC},
the UCL \ac{WEISS},
the UCL \ac{INM}, and
the \ac{BMEIS} at King's College London.
Additionally, I was invited to present the library at multiple events organized by Meta: the PyTorch Ecosystem Day 2021%
\fnurl{https://pytorch.org/ecosystem/pted/2021},
the PyTorch Community Voices Webinar Series%
\fnurl{https://www.youtube.com/watch?v=UEUVSw5-M9M},
and the PyTorch Developer Day 2021%
\fnurl{https://pytorch.org/blog/pytorch-developer-day-2021/}.
TorchIO was also disseminated at the \textit{LATAM Minds} podcast from the Latin American School of Artificial Intelligence%
\fnurl{https://open.spotify.com/episode/03gYyFmwWmiY38U30xJ0Zp?si=4bd8e97ca1a74685}
and within an entry to the MICCAI Educational Challenge 2020%
\fnurl{https://github.com/fepegar/miccai-educational-challenge-2020}.

Outside academia, the presented work paves the way for a data-driven paradigm to diagnose and treat epilepsy.
Our methods may allow clinicians to use data-driven approaches for clinical decision support, as we overcome the scarcity of annotated data in clinical settings.
Our work on classification of seizure videos could be further developed into a product, such as a mobile phone application, to analyze seizures outside the hospital.
This application could be used a clinical decision support tool to give patients higher priority for surgery if their seizures imply a higher risk of death.
The developed software for visualization of seizure semiology may be used for a faster and more objective planning of electrodes implantation and resective surgery.


% COLLABORATION with France, Italy, UK
% EPISURG, years, # patients, public
% Yale?

% Add repos?



% The statement should describe, in no more than 500 words, how the expertise, knowledge, analysis,
% discovery or insight presented in your thesis could be put to a beneficial use. Consider benefits both
% inside and outside academia and the ways in which these benefits could be brought about.

% The benefits inside academia could be to the discipline and future scholarship, research methods
% or methodology, the curriculum; they might be within your research area and potentially within other
% research areas.

% The benefits outside academia could occur to commercial activity, social enterprise, professional
% practice, clinical use, public health, public policy design, public service delivery, laws, public
% discourse, culture, the quality of the environment or quality of life.

% The impact could occur locally, regionally, nationally or internationally, to individuals, communities
% or organizations and could be immediate or occur incrementally, in the context of a broader field of
% research, over many years, decades or longer.

% Impact could be brought about through disseminating outputs (either in scholarly journals or
% elsewhere such as specialist or mainstream media), education, public engagement, translational
% research, commercial and social enterprise activity, engaging with public policy makers and public
% service delivery practitioners, influencing ministers, collaborating with academics and non-academics
% etc.

% Further information including a searchable list of hundreds of examples of UCL impact outside
% of academia please see Research Impact\fnurl{https://www.ucl.ac.uk/impact/} website. For thousands more examples, please see
% REF2014\fnurl{https://results.ref.ac.uk/(S(trbo3sw0ose3zbhpqrknx4rc))/Results/SelectUoa} website.
