\section{Discussion}

The choice of targets for \ac{iEEG} electrodes implantation would benefit from an objective, data-driven method.
In this work, we present an open-source software tool to visualize regions of the brain with a high probability of being associated with the \ac{EZ}, given a set of observed seizure semiologies.

Our tool is not designed to replace clinicians in planning the \ac{iEEG} implantation or the resective surgery, but to support their decisions using a data-driven approach that represents patterns in the literature intuitively.

In the future, we will improve the generalizability of our framework to improve the compatibility with custom databases and other brain parcellation strategies, such as the Desikan-Killiany atlas \cite{desikan_automated_2006}.
We will also develop an online version of our \ac{SVT} that uses a well-established cloud infrastructure service, such as Azure or the Google Cloud Platform, which would allow for a fast and seamless user experience.
Another potential improvement is adding support to parcellate the brain at loading time using deep learning \cite{li_compactness_2017,perez-garcia_fepegarhighresnet_2019}, according to the requirements of the \ac{API}.
This would spare the user the need to wait for hours before the parcellation is generated, which is typically the case for \ac{GIF} \cite{cardoso_geodesic_2015} or FreeSurfer%
\fnurl{https://surfer.nmr.mgh.harvard.edu/}.

% Use deep learning to compute parcellation

% HTTP API for database

% Cloud version hosted on e.g. Azure

% Add support to pass weights or times of semiologies

% Improve compatibility with FreeSurfer and custom databases
