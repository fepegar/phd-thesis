\section{Discussion}
\label{sec:svt_discussion}

The choice of targets for \ac{iEEG} electrodes implantation would benefit from an objective, data-driven method.
In this work, we present an open-source software tool to visualize regions of the brain with a high probability of being associated with the \ac{EZ}, given a set of observed seizure semiologies.

The \ac{SVT} is not designed to replace clinicians in planning the \ac{iEEG} implantation or the resective surgery, but to support their decisions using a data-driven approach that displays patterns in the literature intuitively and objectively.
Our expectation is that the results of this analysis may increase the number of cerebral areas targeted with \ac{iEEG}, but would not reduce them.
This could be caused by the \ac{SVT} highlighting regions not previously considered due to the experience of the team but which the literature suggests may be implicated.  % added by John

In the future, we will improve the generalizability of our framework to improve the compatibility with custom databases and other brain parcellation strategies, such as the Desikan-Killiany atlas \cite{desikan_automated_2006}.
Additional development of a full online version of our \ac{SVT} hosted on a well-established cloud infrastructure service, such as Microsoft Azure or Google Cloud would allow for a fast and seamless user experience.
Another potential improvement is adding support to parcellate the brain at loading time using a deep learning model \cite{li_compactness_2017,perez-garcia_fepegarhighresnet_2019}, that satisfies the requirements of the selected database and corresponding \ac{API}.
This would spare the user the need to wait for hours before the parcellation is generated, which is typically the case for \ac{GIF} \cite{cardoso_geodesic_2015} or FreeSurfer%
\fnurl{https://surfer.nmr.mgh.harvard.edu/}.
