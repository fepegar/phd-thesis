\section{Introduction}

\Acp{AED} are normally used to treat epilepsy.
One third of epilepsies are drug-resistant \cite{engel_what_2016}.
Curative resective surgery can be performed to remove the \ac{EZ} (\cref{chap:resection}).
The location of the \ac{EZ}, ``the area of cortex indispensable for the generation of clinical seizures'' \cite{rosenow_presurgical_2001}, is normally inferred by a multidisciplanary team following non-invasive evaluation such as video-\ac{EEG}, \ac{MRI} or neuropsychological tests.
If the information regarding the location of the \ac{EZ} is discordant between the different tests, \ac{iEEG} electrodes may be implanted to localize it precisely.
To determine the brain regions that need to be recorded, i.e., the targets for the \ac{iEEG} electrodes, the team leverages information from the non-invasive examinations, including seizure semiology from the recorded videos (\cref{chap:videos}).
The choice of targets is therefore influenced by the team's subjective experience and knowledge of the literature.
This leads to substantial variations of implantation strategies across different epilepsy centers \cite{tufenkjian_seizure_2012}.
The diagnostic pathway for surgical planning would benefit from an objective decision support tool to aid clinicians in deducing the \ac{EZ} location from seizure semiology.

One way to plan the implantation objectively is developing a data-driven, automated method.
We performed a systematic literature review to generate an open-access database containing 11230 datapoints, where each datapoint represents a patient with epilepsym showing a certain semiology who became seizure-free after resection of a certain brain structure \cite{alim-marvasti_probabilistic_2021}.
In this context, seizure-free refers to one full year without seizures after resective surgery.
The database, called \textit{Semio2Brain}, can be queried to quantitatively map observed seizure semiologies to brain structures that were removed leading to seizure freedom.
\Cref{tab:single_semiology} shows a conceptual usage example.

\begin{table}
  \setlength{\tabcolsep}{3pt}
  \centering
  \caption[Short]{
    Long.
  }
  \label{tab:single_semiology}
  \begin{tabular}{l*3c}
    \toprule
                          & \textbf{Structure 1} & \textbf{Structure 2} & \textbf{Structure 3} \\
    \midrule
    \textbf{Head version} &                    0 &                    5 &                   20 \\
  \end{tabular}
\end{table}

Show example of multiple semiologies and aggregation

Details on the methods used for aggregation are out of the scope of this thesis

Convenient for clinicians who do not want to code

3D is useful, cite Mark Nowell

Generates a probability map of the epileptogenic zone.

It can be used to support clinical decisions during surgical planning.

Visualization

Automatic planning?
