\section{Introduction}

\Acp{AED} are normally used to treat epilepsy.
One third of epilepsies are drug-resistant \cite{engel_what_2016}.
Curative resective surgery can be performed to remove the \ac{EZ} (\cref{chap:resection}).
The location of the \ac{EZ}, ``the area of cortex indispensable for the generation of clinical seizures'' \cite{rosenow_presurgical_2001}, is normally inferred by a multidisciplanary team following non-invasive evaluation such as video-\ac{EEG}, \ac{MRI} or neuropsychological tests.
If the information regarding the location of the \ac{EZ} is discordant between the different tests, \ac{iEEG} electrodes may be implanted to localize it precisely.
To determine the brain regions that need to be recorded, i.e., the targets for the \ac{iEEG} electrodes, the team leverages information from the non-invasive examinations, including seizure semiology from the recorded videos (\cref{chap:videos}).
The choice of targets is therefore influenced by the team's subjective experience and knowledge of the literature.
This leads to substantial variations of implantation strategies across different epilepsy centers \cite{tufenkjian_seizure_2012}.
The diagnostic pathway for surgical planning would benefit from an objective decision support tool to aid clinicians in deducing the \ac{EZ} location from seizure semiology.

One way to plan the implantation objectively is developing a data-driven, automated method.
We performed a systematic literature review to generate an open-access database containing ``11230 datapoints from 4643 patients across 309 articles'', where each datapoint represents a patient showing a certain semiology who became seizure-free after resection of a certain brain structure \cite{alim-marvasti_probabilistic_2021}.
In this context, seizure-free refers to one full year without seizures after resective surgery.
The database, called \svtdatabase, can be queried to quantitatively map observed seizure semiologies to brain structures that were removed leading to seizure freedom.
The brain structures included in \svtdatabase are extracted from the Neuromorphometrics atlas%
\fnurl{http://www.neuromorphometrics.com}.
\Cref{tab:single_semiology} shows a conceptual usage example.

\begin{table}
  \setlength{\tabcolsep}{3pt}
  \centering
  \caption[Result of querying an imaginary database with one semiology]{
    Result of querying an imaginary database with the semiology `Head version' and exemplar brain structures A, B and C.
    In this example, 20 patients presenting `Head version' become seizure-free after removal of structure C according to the literature, suggesting a high probability of the \ac{EZ} being associated with that structure.
    Five patients became seizure-free after resection of structure B, and none after resection of structure A.
    This result would support the decision of implanting electrodes in structure C (and maybe B), as they are likely to be associated with the \ac{EZ} according to the retrospective information in the literature.
    The actual list of brain structures depends on the method used to parcellate the brain.
  }
  \label{tab:single_semiology}
  \begin{tabular}{l*3c}
    \toprule
                          & \textbf{Structure A} & \textbf{Structure B} & \textbf{Structure C} \\
    \midrule
    \textbf{Head version} &                    0 &                    5 &                   20 \\
  \end{tabular}
\end{table}

Researchers who do not wish to code would benefit from a \ac{GUI} to query the database.
Moreover, visualizing the probability of each structure containing the \ac{EZ} on \ac{3DMMI} could help plan the resection and \ac{iEEG} implantation strategies \cite{nowell_resection_2017, nowell_utility_2015}, potentially using \ac{ATP} \cite{sparks_automated_2017}.
Finally, a \ac{3DMMI} visualization could be used to perform qualitative and quantitative analyses of the retrospective information contained in the \svtdatabase.

In this work, we present our software tool to visualize the \ac{EZ} probability map on \ac{3DMMI}.

% Show example of multiple semiologies and aggregation
% Details on the methods used for aggregation are out of the scope of this thesis
