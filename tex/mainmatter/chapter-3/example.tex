\section{Example of clinical usage}

% Pt # 771 (0999, roaj) This patient was a 28-year-old right-handed gentleman assessed for epilepsy surgery having had nocturnal generalised seizures since the age of 12 and subsequently developed stereotyped head and eye version to the right with body turning, tonic left leg extension and raising of the left arm with speech arrest. Ictal EEG was non lateralising and interictal EEG showed bitemporal sharp waves. MRI showed cortical dysplasia in the left superior frontal gyrus and Ictal SPECT highlighted superior more than middle frontal gyrus; intracranial EEG was performed to map eloquent cortex and a left frontal resection including the supplementary motor cortex resulted in complete seizure-freedom for 4 years (ILAE 1 for four years of follow up).

In this section, we demonstrate a usage example of our \ac{SVT} with a retrospective case of a patient who underwent epilepsy resective surgery at the \ac{NHNN} (Queen Square, London, UK).
We use the \texttt{mega\_analysis} querying module, which uses the \svtdatabase database \cite{alim-marvasti_mapping_2021,alim-marvasti_probabilistic_2021}.

The patient was right-handed and presented head version to the right at the beginning of seizures.
Ictal \ac{EEG} was non lateralizing and interictal \ac{EEG} showed bitemporal sharp waves.
We used the patient's preoperative \ac{T1w} \ac{MRI} as reference for the visualization.
The Neuromorphometrics brain parcellation was generated using \ac{GIF} \cite{cardoso_geodesic_2015}.

We queried the database using the semiology \textit{Head version (right)} and setting the dominant hemisphere to \textit{Left}.
The most highlighted structures concentrate in the left frontal lobe (\cref{fig:svt_case_heatmap}).

\begin{figure}
  \centering
  \svtscreenshot{svt_case_heatmap}
  \caption[Querying the database using a retrospective real case]{
    Querying the database using data from a retrospective real case.
    The selected semiology term was \textit{Head version (right)}, and the dominant hemisphere was \textit{Left}.
    The regions with highest numbers of datapoints concentrate around the left frontal lobe (represented on the right side of the axial and coronal views).
  }
  \label{fig:svt_case_heatmap}
\end{figure}


TALK ABOUT THE COMPARISON

\begin{figure}
  \centering
  \svtscreenshot{svt_resection}
  \caption[Comparison of the probability map and the postoperative MRI]{
    Qualitative comparison of the \ac{EZ} probability map and the postoperative \ac{MRI}.
    Note the overlap between the highlighted structures and the resection cavity.
    The images were rigidly registered, and the crosshairs are centered on approximately the same brain region.
    The patient became seizure-free after resective surgery.
  }
  \label{fig:svt_resection}
\end{figure}

% Left dominant hemisphere
% Aphasia
% Eye version (right)
% Head or body turn (right)
% Tonic (left)


Describe case

Use patient's data (on MNI)

Pass semiologies etc

Visualize heatmap

Show resection in patient's space and say it was seizure-free

Show preop with lesion in patient's space
