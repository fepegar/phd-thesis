\subsection{Resection simulation for self-supervised learning}
\label{sec:simulation}

\newcommand{\AAA}{\bm{A}}
\newcommand{\NN}{\mathcal{N}}


$\phi\simul$ takes images from $\Dom\pre$ to generate training instances by simulating a realistic shape, location and intensity pattern for the \ac{RC}.
We present simulation of cavity shape and label in \cref{sec:cavity,sec:cavity_constrain}, respectively.
In \cref{sec:texture_cavity}, we present our method to generate the resected image.


\subsubsection{Initial cavity shape}
\label{sec:cavity}

To simulate a realistic \ac{RC}, we consider its topological and geometric properties: it is a single volume with a non-smooth boundary.
We generate a geodesic polyhedron with frequency $f$ by subdividing the edges of an icosahedron $f$ times and projecting each vertex onto a parametric sphere with a unit radius centered at the origin.
This polyhedron models a spherical surface $S = \{ V, F \}$ with vertices
$
  V = \left\{
    \vv_i \in \R^3
  \right\}
  _{i = 1}^{n_V}
$
and faces
$
  F = \left\{
    \bm{f}_k \in \mathbb{N}^3
  \right\}
  _{k = 1}^{n_F}
$, where $n_V$ and $n_F$ are the number of vertices and faces, respectively.
%
Each face $\bm{f}_k = \{ i_1^k, i_2^k, i_3^k \}$ is a sequence of three non-repeated vertex indices.

To create a non-smooth surface, $S$ is perturbed with simplex noise \cite{perlin_improving_2002}, a procedural noise generated by interpolating pseudorandom gradients on a multidimensional simplicial grid.
We chose simplex noise as it simulates natural-looking textures or terrains and is computationally efficient for multiple dimensions.
The noise $\eta : \R^3 \to [-1, 1]$ at point $\p \in \R^3$ is a weighted sum of the noise contribution for $\omega$ different octaves, with weights $\{\gamma ^ {n - 1}\}_{n = 1}^{\omega}$ controlled by the persistence parameter $\gamma$.
The displacement $\delta$ of a vertex $\vv$ is:
\begin{equation}
  \delta(\vv)
  = \eta \left( \frac{\vv + \bm{\mu} }{\zeta}, \omega, \gamma \right)
\end{equation}
where
$\zeta$ is a scaling parameter to control smoothness
and $\bm{\mu}$ is a shifting parameter that adds stochasticity
(equivalent to a random number generator seed).
%
Each vertex $\vv_i$ is displaced radially to create a perturbed sphere:
$
V_{\delta}
  = \left\{
  \vv_i
  + \delta(\vv_i)
  \frac{\vv_i}{\|\vv_i\|}
  \right\}
  _{i = 1}^{n_V}
  = \left\{
  \vv_{\delta i}
  \right\}
  _{i = 1}^{n_V}
$.

Next, a series of transforms is applied to $V_{\delta}$ to modify the mesh's volume and shape.
To add stochasticity, random rotations around each axis are applied to $V_{\delta}$ with the rotation transform
$T\st{R}(\bm{\theta}\st{r}) = R_x(\theta_x) \circ R_y(\theta_y) \circ R_z(\theta_z)$,
where~$\circ$~indicates a transform composition and
$R_i(\theta_i)$ is a rotation of $\theta_i$ radians around axis $i$.
$T\st{S}(\bm{r})$ is a scaling transform,
where $(r_1, r_2, r_3) = \bm{r}$ are semiaxes of an ellipsoid
with volume $v$ used to model the cavity shape.
The semiaxes are computed as
$r_1 = r$, $r_2 = \lambda r$ and $r_3 = r /\lambda$,
where $r = (3 v / 4)^{1/3}$ and
$\lambda$ controls the semiaxes length ratios\footnote{
  Note the volume of an ellipsoid with semiaxes $(a, b, c)$ is $v = \frac{4}{3} \pi a b c$.
}.
These transforms are applied to $V_{\delta}$ to define the initial resection cavity surface $S\st{E} = \{ V\st{E}, F \}$, where
$V\st{E} =
\{
  T\st{S}(\bm{r})
  \circ T\st{R}(\bm{\theta}\st{r})(
    \vv_{\delta i})
\}
_{i = 1}^{n_V}
$.


\subsubsection{Cavity label}
\label{sec:cavity_constrain}


\begin{figure}
  \centering
  % \captionsetup[subfigure]{justification=centering}
  \begin{subfigure}{0.3\textwidth}
    \includegraphics[width=0.8\linewidth]{Ma}
    \caption{$S_a$ on $\M\st{GM}^h$\label{fig:sama}}
  \end{subfigure}
  \begin{subfigure}{0.3\textwidth}
    \includegraphics[width=0.8\linewidth]{Mb}
    \caption{$S_a$ on $\M\st{R}^h$\label{fig:samb}}
  \end{subfigure}
  \begin{subfigure}{0.3\textwidth}
    \includegraphics[width=0.8\linewidth]{Mr}
    \caption{$\Y\simul = \M_{S_a} \odot \M\st{R}^h$\label{fig:mr}}
  \end{subfigure}

  \caption{%
    Simulation of the ground-truth cavity label.
    $S_a$ (blue) is computed by centering $S\st{E}$ on $\bm{a}$, a random positive voxel (red) of $\M\st{GM}^h$ (\subref{fig:sama}).
    $\M_{S_a}$ is a binary mask derived from $S_a$.
    $\Y\simul$ (\subref{fig:mr}) is the intersection of $\M_{S_a}$ and $\M\st{R}^h$ (\subref{fig:samb}).
  }
  \label{fig:shape}
\end{figure}



The simulated \ac{RC} should not span both hemispheres or include extracerebral tissues such as bone or scalp.
This section describes our method to ensure that the \ac{RC} appears in anatomically plausible regions.

A \ac{T1w} \ac{MRI} is defined as $\X\pre : \Omega \to \R$.
A full brain parcellation $\bm{P} : \Omega \to Z$ is generated \cite{cardoso_geodesic_2015} for $\X\pre$,
where $Z$ is the set of segmented structures.
A cortical gray matter mask $\M\st{GM}^h : \Omega \to \{0, 1\}$
of hemisphere $h$ is extracted from $\bm{P}$,
where $h$ is randomly chosen from $H = \{\text{left}, \text{right}\}$ with equal probability.

A ``resectable hemisphere mask'' $\M\st{R}^h$ is generated from $\bm{P}$ and $h$ such that $\M\st{R}^h (\p) = 1$ if
${\bm{P}(\p) \neq \{M\st{BG}, M\st{BT}, M\st{CB}, M_{\hat{h}} \} }$
and $0$ otherwise,
where $M\st{BG}$, $M\st{BT}$, $M\st{CB}$ and $M_{\hat{h}}$ are the labels in $Z$ corresponding to the background, brainstem, cerebellum and contralateral hemisphere, respectively.
$\M\st{R}^h$ is smoothed using a series of binary morphological operations, for realism.


A random voxel $\bm{a} \in \Omega$ is selected such that $\M\st{GM}^h(\bm{a}) = 1$.
A translation transform $T\st{T}(\bm{a} - \bm{c})$ is applied to $S\st{E}$ so $S_a = T\st{T}(\bm{a} - \bm{c}) (S\st{E})$ is centered on $\bm{a}$.

A binary image $\binimg{\M_{S_a}}$ is generated from $S_a$ such that $\M_{S_a}(\p) = 1$ for all $\p$ within $S_a$ and $\M_{S_a}(\p) = 0$ outside.
Finally, $\M_{S_a}$ is restricted by $\M\st{R}^h$ to generate the cavity label $\Y\simul = \M_{S_a} \odot \M\st{R}^h$, where $\odot$ represents the Hadamard product.
\cref{fig:shape} illustrates the process.



\subsubsection{Simulating cavities filled with CSF}
\label{sec:texture_cavity}

Brain \acp{RC} are typically filled with \ac{CSF}.
To generate a realistic \acs{CSF} texture,
we create a ventricle mask
${\M\st{V} : \Omega \to \{ 0, 1 \}}$ from $\bm{P}$, such that
$\M\st{V}(\p) = 1$ for all $\p$ within the ventricles and
$\M\st{V}(\p) = 0$ outside.
Intensity values within the ventricles are assumed to have
a normal distribution \cite{gudbjartsson_rician_1995}
with a mean $\mu\st{CSF}$ and standard deviation $\sigma\st{CSF}$
calculated from voxel intensity values in
$\{ \X\pre(\p) \mid \p \in \Omega \land \M\st{V}(\p) = 1 \}$.
A \acs{CSF}-like image is then generated as $\X\st{CSF}(\p) \sim \NN (\mu\st{CSF}, \sigma\st{CSF}), \forall \p \in \Omega$.


We use $\Y\simul$ to guide blending of $\X\st{CSF}$ and $\X\pre$ as follows.
A Gaussian filter is applied to $\Y\simul$ to obtain a smooth alpha channel $\img{\AAA_\alpha}{[0, 1]}$ defined as
$
  \AAA_\alpha
  = \Y\simul
  * \bm{G}_{\NN}(\bm{\sigma}),
$
where
$*$ is the convolution operator
and $\bm{G}_{\NN}(\bm{\sigma})$ is a 3D Gaussian kernel with standard deviations
$\bm{\sigma} = (\sigma_x, \sigma_y, \sigma_z)$.
Then, $\X\st{CSF}$ and $\X\pre$ are blended by the convex combination
\begin{equation}
  \X\simul
  = \AAA_\alpha \odot \X\st{CSF}
  + (1 - \AAA_\alpha) \odot \X\pre
\end{equation}

We use $\bm{\sigma} > 0$ to mimic partial-volume effects at the cavity boundary.
The blending process is illustrated in \cref{fig:texture}.


\begin{figure}
  \centering
  \captionsetup[subfigure]{aboveskip=3pt, belowskip=5pt}

  \begin{subfigure}{0.15\textwidth}
    \includegraphics[width=0.99\linewidth]{texture_mri}
    \caption{\label{fig:tmri}}
  \end{subfigure}
  \begin{subfigure}{0.15\textwidth}
    \includegraphics[width=0.99\linewidth]{texture_checkerboard}
    \caption{\label{fig:checkerboard}}
  \end{subfigure}
  \begin{subfigure}{0.15\textwidth}
    \includegraphics[width=0.99\linewidth]{Mr}
    \caption{\label{fig:tmh}}
  \end{subfigure}
  \begin{subfigure}{0.15\textwidth}
    \includegraphics[width=0.99\linewidth]{texture_hard}
    \caption{\label{fig:blh}}
  \end{subfigure}
  \begin{subfigure}{0.15\textwidth}
    \includegraphics[width=0.99\linewidth]{texture_mask_soft}
    \caption{\label{fig:tms}}
  \end{subfigure}
  \begin{subfigure}{0.15\textwidth}
    \includegraphics[width=0.99\linewidth]{texture_soft}
    \caption{\label{fig:bls}}
  \end{subfigure}

  \caption{
    Simulation of resected image $\X\simul$.
    We use a checkerboard for visualization.
    Two scalar-valued images $\X\pre$ (\subref{fig:tmri})
    and $\X_2$ (\subref{fig:checkerboard})
    are blended using $\Y\simul$ (\subref{fig:tmh})
    and $\sigma_i = \SI{0}{\milli \meter}$ to create an image with hard boundaries (\subref{fig:blh})
    and $\sigma_i = \SI{5}{\milli \meter}$ (\subref{fig:tms})
    for an image with soft boundaries (\subref{fig:bls}),
    mimicking partial-volume effects.
  }
  \label{fig:texture}
\end{figure}
