\section{Discussion}

We have presented TorchIO, a new library to efficiently load, preprocess,
augment and sample medical imaging data during the training of \acp{CNN}.
%
It is designed in the style of the deep learning framework PyTorch
to provide medical imaging specific preprocessing and data augmentation
algorithms.


The main motivation for developing TorchIO as an open-source toolkit is to help
researchers standardize medical image processing pipelines and allow them to
focus on the deep learning experiments.
%
It also encourages good open-science practices, as it supports experiment
reproducibility and is version-controlled so that the software can be cited
precisely.


The library is compatible with other higher-level deep learning frameworks for
medical imaging such as \ac{MONAI}.
%
For example, users can benefit from TorchIO's \ac{MRI} transforms and
patch-based sampling while using \ac{MONAI}'s networks, losses, training pipelines
and evaluation metrics.

\textcolor{rev2}{%
The main limitation of TorchIO is that most transforms are not differentiable.
%
The reason is that PyTorch tensors stored in TorchIO data
structures must be converted to SimpleITK images or NumPy arrays
within most transforms, making them not compatible
with PyTorch's automatic differentiation engine.
%
However, compatibility between PyTorch and ITK has recently been
improved, partly thanks to the appearance of the \ac{MONAI}
project \cite{mccormick_itk_2021}.
%
Therefore, TorchIO might provide differentiable transforms in the future, which
could be used to implement, e.g., spatial transformer networks for image
registration \cite{lee_image-and-spatial_2019}.
%
Another limitation is that many more transforms that are \ac{MRI}-specific exist than for other imaging modalities such as \ac{CT} or \ac{US}.
This is in part due to more users working on \ac{MRI} applications and requesting \ac{MRI}-specific transforms.
However, we welcome contributions for other modalities as well. 
}


In the future, we will work on extending the preprocessing and augmentation
transforms to different medical imaging modalities
such as \ac{CT} or \ac{US}, and improving compatibility with related works.
%
The source code, as well as examples and documentation,
are made publicly available online, on GitHub.
%
We welcome feedback, feature requests, and contributions to the library,
either by creating issues on the GitHub repository or by emailing the authors.