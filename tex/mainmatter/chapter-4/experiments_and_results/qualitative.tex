\subsubsection{Qualitative evaluation on brain tumor resection dataset}

We used the \ac{BITE} dataset \cite{mercier_online_2012} to evaluate the ability of our self-supervised model to segment resection cavities on images from a different institution, modality and pathology than the datasets used for quantitative evaluation.
For postprocessing, all but the largest binary connected component were removed.
The model successfully segmented the resection cavity on 11/13 images, even though some contained challenging features (\cref{fig:bite}).

\newcommand{\qualit}[1]{\includegraphics[width=0.14\linewidth]{bite/cropped/#1}}
\newcommand{\qualitfig}[1]{
  \centering
  \qualit{#1_sag}%
  \enskip
  \qualit{#1_seg_sag}%
  \quad
  \qualit{#1_cor}%
  \enskip
  \qualit{#1_seg_cor}%
  \quad
  \qualit{#1_axi}%
  \enskip
  \qualit{#1_seg_axi}
}

% https://tex.stackexchange.com/a/381477/216202
\begin{figure}
  \centering
  \begin{subfigure}{\textwidth}
    \qualitfig{2_low_contrast}
    \caption{Image with low contrast between the cavity and the brain \label{fig:bite_low_contrast}}
  \end{subfigure}
  \vskip\baselineskip
  \begin{subfigure}{\textwidth}
    \qualitfig{4_air}
    \caption{Image with air and \ac{CSF} within the cavity \label{fig:bite_air}}
  \end{subfigure}
  \vskip\baselineskip
  \begin{subfigure}{\textwidth}
    \qualitfig{5_aniso}
    \caption{Image with a highly anisotropic voxel spacing \label{fig:bite_aniso}}
  \end{subfigure}
  \vskip\baselineskip
  \begin{subfigure}{\textwidth}
    \qualitfig{12_motion}
    \caption{Image with \ac{MRI} motion artifacts and adjacent edema \label{fig:bite_motion}}
  \end{subfigure}
  \caption[Qualitative evaluation on postoperative brain tumor data]{
    Qualitative evaluation of the self-supervised model on a dataset of postoperative brain tumor \acf{T1wCE} \ac{MRI}.
    The model is robust to multiple challenging scenarios:
    low contrast between the cavity and the brain (\subref{fig:bite_low_contrast}),
    air and \acf{CSF} within the resection cavity (\subref{fig:bite_air}),
    highly anisotropic voxel spacing (\subref{fig:bite_aniso}),
    motion artifacts and edema (\subref{fig:bite_motion}),
    and a different modality than used for training (all).
    Note that these images are from a different institution, modality and pathology than the datasets used for quantitative evaluation.
    Ground-truth labels are not shown as manual annotations are not available.
  }
  \label{fig:bite}
\end{figure}




\subsubsection{Qualitative evaluation on intraoperative image}

We used our baseline model to segment the resection cavity on one intraoperative \ac{MRI} from our institution.
Despite the large domain shift between the training dataset and the intraoperative image, which includes a retracted skin flap and a missing bone flap, the model was able to correctly estimate the resection cavity, discarding similar regions filled with \ac{CSF} or air (\cref{fig:intra}).

\begin{figure}
  \centering
  \includegraphics[trim=0 0 0 12, clip, width=\linewidth]{intra}
  \caption[Qualitative result on an intraoperative \acs{MRI}]{
    Qualitative result on an intraoperative \ac{MRI}.
    The baseline model correctly discarded regions filled with air or \acf{CSF} outside of the resection cavity.
  }
  \label{fig:intra}
\end{figure}
