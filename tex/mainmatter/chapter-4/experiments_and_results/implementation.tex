\subsection{Network architecture and implementation details}

We used the PyTorch deep learning framework \cite{paszke_pytorch_2019}, training with \ac{AMP} on two 32-GB TESLA V100 \acp{GPU}.
In the \ac{AMP} setting, some operations such as convolution are computed using half precision (i.e., 16-bit floating point) to reduce the computational burden while maintaining a similar performance.

We implemented a variant of 3D U-Net \cite{cicek_3d_2016} using two downsampling and upsampling blocks, upsampling with trilinear interpolation for the synthesis path, and 1/4 of the filters for each convolutional layer.
We used dilated convolutions \cite{chen_deeplab_2017}, starting with a dilation factor of one, then increased or decreased in steps of one after each downsampling or upsampling block, respectively.
This results in a model with the same receptive field (a cube of length 88 mm) but $\approx 77 \times$ fewer parameters (\num{246156}) than the original 3D U-Net, reducing overfitting and computational burden.

Convolutional layers were initialized using He's method and followed by batch normalization and nonlinear \ac{PReLU} activation functions \cite{ioffe_batch_2015,he_delving_2015}.
We used adaptive moment estimation (AdamW) \cite{kingma_adam_2014,loshchilov_decoupled_2019} to adjust the learning rate during training, with weight decay of~$10^{-2}$ and a learning scheduler that divides the learning rate by ten every 20 epochs.
We optimized our network to minimize the mean soft Dice loss \cite{milletari_v-net_2016} of each mini-batch, for all the experiments.
A mini-batch size of ten images (five per \ac{GPU}) was used for training.
Self-supervised training took about 27 hours.
Fine-tuning on a small annotated dataset took about seven hours.

We used Sacred \cite{greff_sacred_2017} and TensorBoard \cite{abadi_tensorflow_2016} to configure, log and visualize our experiments.
