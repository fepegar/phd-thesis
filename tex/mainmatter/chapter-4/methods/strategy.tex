\subsection{Learning strategy}
\label{sec:learning_strategy}


\subsubsection{Problem statement}

\newcommand{\loss}{\mathcal{L}}
\newcommand{\expec}{\mathbb{E}}
\newcommand{\exppost}{\expec_{\Dom\post}}

The overall objective is to automatically segment \acp{RC} from postoperative \ac{T1w} \ac{MRI} using a \ac{CNN} $f_{\bm{\theta}}$ parameterized by weights $\bm{\theta}$.
Let $\X_{\text{post}} : \Omega \to \R$ and $\Y\cav  : \Omega \to \{ 0, 1 \}$ be a postoperative \ac{T1w} \ac{MRI} and its cavity segmentation label, respectively, where $\Omega \subset \R^3$.
$\X_{\text{post}}$ and $\Y_{\text{cavity}}$ are drawn from the data distribution $\Dom\post$.
In model training, the aim is to minimize the expected discrepancy between the label $\Y\cav$ and network prediction $f_{\bm{\theta}}(\X\post)$.
Let $\loss$ be a loss function that estimates this discrepancy (e.g., Dice loss).
The optimization problem for the network parameters $\bm{\theta}$ is:
\begin{equation}
  \bm{\theta}^* =
  \argmin_{\bm{\theta}}
  \exppost \left[
    \loss \left(
      f_{\bm{\theta}} \left( \X\post \right),
      \Y\cav
    \right)
  \right]
  \label{eq:problem_optimization}
\end{equation}


In a fully-supervised setting, a labeled dataset $D\post = \{ (\X_{\text{post}_i}, \Y_{\text{cavity}_i}) \}_{i = 1}^{n\post}$ is employed to estimate the expectation defined in \cref{eq:problem_optimization} as:
\begin{equation}
  \exppost \left[
    \loss \left(
      f_{\bm{\theta}} \left( \X\post \right), \Y\cav
    \right)
  \right]
  \approx \frac{1}{n\post} \sum_{i=1}^{n\post} \loss(f_{\bm{\theta}}(\X_{\text{post}_i}), \Y_{\text{post}_i})
  \label{eq:problem_optimization_fully}
\end{equation}

In practice, \acp{CNN} typically require an annotated dataset with a large $n\post$ to generalize well for unseen instances.
However, given the time and expertise required to annotate scans, $n\post$ is often small.
We present a method to artificially increase $n\post$ by simulating postoperative \acp{MRI} and associated labels from preoperative scans.


\subsubsection{Simulation for domain adaptation and self-supervised learning}
\label{sec:sim_res_self}

Let $D\pre = \{ \X_{\text{pre}_i} \}_{i = 1}^{n\pre}$ be a dataset of preoperative \ac{T1w} \ac{MRI}, drawn from the data distribution $\Dom\pre$.
We propose to generate a simulated postoperative dataset $D\simul = \{ (\X_{\text{sim}_i}, \Y_{\text{sim}_i}) \}_{i = 1}^{n\simul}$ using the preoperative dataset $D\pre$.
Specifically, we aim to build a generative model $\phi\simul : \X\pre \mapsto (\X\simul, \Y\simul)$ that transforms preoperative images into simulated, annotated postoperative images that imitate instances drawn from the postoperative data distribution $\Dom\post$.
$D\simul$ can then be used to estimate the expectation in \cref{eq:problem_optimization}:
\begin{equation}
  \exppost \left[\
    \loss\left(
      f_{\bm{\theta}} \left(\X\post \right), \Y\cav \right)
    \right]
    \approx \frac{1}{n\simul}\sum_{i=1}^{n\simul} \loss(f_{\bm{\theta}}(\X_{\text{sim}_i}),  \Y_{\text{sim}_i})
  \label{eq:problem_optimization_sim}
\end{equation}

Simulated images can be generated from any unlabeled preoperative dataset.
Therefore, the size of the simulated dataset can be much greater than the annotated dataset $D\post$, i.e., $n\simul\gg n\post$.
The network parameters $\bm{\theta}$ can be optimized by minimizing \cref{eq:problem_optimization_sim} using stochastic gradient descent, leading to a trained predictive function $f_{\bm{\theta_}{\text{sim}}}$.
Finally, $f_{\bm{\theta_}{\text{sim}}}$ can be fine-tuned on $D\post$ to improve performance on the postoperative domain $\Dom\post$.
