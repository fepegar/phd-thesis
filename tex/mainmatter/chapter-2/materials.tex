\subsection{Video acquisition}

Patients were recorded using two full high-definition ($1920 \times 1080$ pixels, 30 \ac{FPS}) cameras installed in the \ac{EMU} as part of standard clinical practice.
Infrared is used for acquisition in scenes with low light intensity, such as during nighttime.
The acquisition software (Micromed, Treviso, Italy) automatically resizes one of the video streams ($800 \times 450$), superimposes it onto the top-left corner of the other stream and stores the montage using MPEG-2.
See the supplementary materials for six examples of videos in our dataset.


\subsection{Dataset description and ground-truth definitions}
\label{sec:dataset}

A neurophysiologist (A.A.) annotated for each seizure the following times: clinical seizure onset $t_0$, onset of the clonic phase $t_G$ (\acp{TCS} only) and clinical seizure offset $t_1$.
The annotations were confirmed using \ac{EEG}.

We curated a dataset comprising 141 \acp{FOS} and 77 \acp{TCS} videos from 68 epileptic patients undergoing presurgical evaluation at the National Hospital for Neurology and Neurosurgery, London, United Kingdom.
As patients with only generalized onset seizures are typically not considered for surgery \cite{duncan_brain_2016}, our dataset does not contain any seizure of this type.  % originally commented out
To reduce the seizure class imbalance, we discarded seizures where $t_1 - t_0 < \SI{15}{\second}$, as this threshold is well under the shortest reported time for \acp{TCS} \cite{jenssen_how_2006}.
After discarding short videos, which accounted for 1\% of the duration of the initial dataset, there were 106 \acp{FOS}.
The `median (min, max)' number of seizures per patient is 2 (1, 16).
The duration of \ac{FOS} and \ac{TCS} is 53 (16, 701) s and 93 (51, 1098) s, respectively.
The total duration of the dataset is 298 minutes, 20\% of which correspond to \ac{TCS} phase (i.e., the time interval $[t_G, t_1]$).
Two patients had only \ac{FOS}, 32 patients had only \ac{TCS}, and 34 had seizures of both types.
The `mean (standard deviation)' of the percentage of the seizure duration before the appearance of generalizing semiology, i.e., $r = (t_G - t_0) / (t_1 - t_0)$, is 0.56 (0.18), indicating that patients typically present generalizing semiological features in the second half of the seizure.

Let a seizure video be a sequence of $K$ frames starting at $t_0$.
Let the time of frame $k \in \{ 0, \dots, K - 1 \}$ be ${t_k = t_0 + \frac{k}{f}}$, where $f$ is the video frame rate.
We use 0 and 1 to represent \ac{FOS} and \ac{TCS} labels, respectively.
The ground-truth label $y_k \in \{0, 1\}$ for frame $k$ is defined as $y_k \coloneqq 0$ if $t_k < t_G$ and 1 otherwise, where $t_G \rightarrow \infty$ for \acp{FOS}.

Let $\x \in \R ^ {3 \times l \times h \times w}$ be a stack of frames or \textit{snippet}, where
$3$ denotes the RGB channels,
$l$ is the number of frames,
and $h$ and $w$ are the number of rows and columns in a frame, respectively.
The label for a snippet starting at frame $k$ is
\begin{equation}
  Y_k \coloneqq
  \left\{
    \begin{array}{ll}
      0 & \mbox{if } \frac{t_k + t_{k + l}}{2} < t_G \\
      1 & \mbox{otherwise}
    \end{array}
  \right.
\end{equation}
