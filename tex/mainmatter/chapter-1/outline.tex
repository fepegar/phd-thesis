\section{Thesis outline}

This thesis is structured in six chapters.
The order of \cref{chap:videos,chap:svt,chap:resection} follows the temporal order of the associated elements of the clinical pathway.
The literature review related to each topic is presented in the corresponding chapters.

In the \textbf{current chapter}, an overview of epilepsy was presented.
Then, the relevant elements in the clinical pathway for the treatment of epilepsy and their corresponding challenges were described.

The \textbf{second chapter} (\nameref{chap:videos}) proposes our open-source framework for automatic classification of seizures from videos: \ac{GESTURES}.

The \textbf{third chapter} (\nameref{chap:svt}) introduces the \ac{SVT}, a piece of software to visualize a probability map of the \ac{EZ} location on neuroimages given a set of observed seizure semiologies, using an evidence-based approach.

The \textbf{fourth chapter} (\nameref{chap:resection}) presents our framework for segmentation of brain resection cavities from postoperative \acp{MRI}.

The \textbf{fifth chapter} (\nameref{chap:torchio}) describes our open-source Python library for data loading, preprocessing and augmentation TorchIO, which was initially developed in the context of the work presented in \cref{chap:resection}.

Finally, the \textbf{sixth chapter} (\nameref{chap:discussion}) concludes the thesis with a summary of the contributions presented in each chapter and outlines potential future research directions.
