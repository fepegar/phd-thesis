\section{Thesis contributions}

% Intro
Epilepsy is a complex disorder with a complicated treatment pathway.
Symptoms are strongly heterogeneous and may sometimes be interpreted differently by teams in different institutions.
Decisions are often taken subjectively, based on clinicians' experience, who have seen a finite number of cases in their careers.
Data-driven techniques, such as machine learning, may be used to take objective decisions based on large amounts of retrospective data.

% Problem statement
The overall goal of this thesis is to identify elements in the current clinical pathway for the treatment of epilepsy that are highly subjective or labor-intensive, and propose data-driven computational methods for clinical decision support.

% Challenges
Deep learning, the main technique utilized in this thesis, has seen great success in the last decade, mostly thanks to improvements in hardware, such as \acp{GPU}, and the vast amount of data generated every day.
However, large datasets are typically not readily available in clinical settings because of privacy concerns and expensive annotation.
% How we're solving them
In this thesis, we leverage techniques such as transfer learning, self-supervised learning and semi-supervised learning to overcome these low-data regimes.


\subsection{Classification of seizure videos}

\subsection{Visualization of seizure semiology in brain images}

\subsection{Segmentation of postoperative brain resection cavities}

\subsection{Publicly available open-source frameworks}
