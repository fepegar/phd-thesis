\section{Thesis outline and contributions}

% Should I merge with outline?

% Intro
Epilepsy is a complex disorder with a complicated treatment pathway.
Symptoms are strongly heterogeneous and may sometimes be interpreted differently by teams in different institutions.
Decisions are often taken subjectively, based on clinicians' experience, who have seen a finite number of cases in their careers.
Data-driven techniques, such as machine learning, may be used to take objective decisions based on large amounts of retrospective data.

% Problem statement
The overall goal of this thesis is to identify elements in the current clinical pathway for the treatment of epilepsy that are highly subjective or labor-intensive, and propose data-driven computational methods for clinical decision support.

% Challenges
Deep learning, the main set of techniques utilized in this thesis, has seen great success in the last decade, mostly thanks to improvements in hardware, such as \acp{GPU}, and the vast amount of data generated every day.
However, large datasets are typically not readily available in clinical settings because of privacy concerns and expensive annotation.
% How we're solving them
In this thesis, we leverage techniques such as transfer learning, self-supervised learning and semi-supervised learning to overcome these low-data regimes.

The document is structured in six chapters.
The order of \cref{chap:videos,chap:svt,chap:resection} follows the temporal order of the associated elements of the clinical pathway.
The literature review related to each topic is presented in the corresponding chapters.


In the \textbf{current chapter}, an overview of epilepsy was first presented.
Then, the relevant elements in the clinical pathway for the treatment of epilepsy and their corresponding challenges were described.


The \textbf{second chapter} (\nameref{chap:videos}) introduces \ac{GESTURES} our open-source framework for automatic classification of seizures from videos.
I present a novel method combining convolutional and recurrent neural networks to model seizures of arbitrary duration.
Our deep learning approach is robust to occlusions by bed linens and clinical staff, differences in illumination and pose, and poor video quality caused by compression artifacts or details out of focus.
The code is available on GitHub%
\fnurl{https://github.com/fepegar/gestures-miccai-2021}.


The \textbf{third chapter} (\nameref{chap:svt}) describes a piece of software developed in collaboration with neurologists Ali Alim-Marvasti and Gloria Romagnoli.
They performed a systematic literature review to generate the Semio2Brain database, which maps seizure semiologies to brain regions \cite{alim-marvasti_probabilistic_2021}.
Ali also wrote software to query the database, which is an Excel spreadsheet.
The input is a set of observed semiologies and some additional optional patient information such as the dominant hemisphere.
My contributions to this project are 1) a 3D Slicer module \cite{fedorov_3d_2012} that reads the output of the querying tool and generates a 3D visualization on a parcellated brain \ac{MRI}, where the brightness associated to each brain structure is proportional to the probability of the \ac{EZ} being in the structure; and 2) the software engineering aspects of the project: \ac{PIP}-compatible Python package for the querying software, \ac{CI}, and an \ac{API} to access the Python package from 3D Slicer.
The database and code are freely available on GitHub%
\fnurl{https://github.com/thenineteen/Semiology-Visualisation-Tool}.
Our \ac{SVT}...  % TODO


The \textbf{fourth chapter} (\nameref{chap:resection}),


The \textbf{fifth chapter} (\nameref{chap:torchio}),


Finally, the \textbf{sixth chapter} (\nameref{chap:conclusion}), concludes the thesis and outlines potential future research directions.
