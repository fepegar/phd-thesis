\section{Thesis outline and contributions}

% Should I merge with outline?

% Intro
Epilepsy is a complex disorder with a complicated treatment pathway.
Symptoms are strongly heterogeneous and may sometimes be interpreted differently by teams in different institutions.
Decisions are often taken subjectively, based on clinicians' experience, who have seen a finite number of cases in their careers.
Data-driven techniques, such as machine learning, may be used to take objective decisions based on large amounts of retrospective data.

% Problem statement
The overall goal of this thesis is to identify elements in the current clinical pathway for the treatment of epilepsy that are highly subjective or labor-intensive, and propose data-driven computational methods for clinical decision support.

% Challenges
Deep learning, the main set of techniques utilized in this thesis, has seen great success in the last decade, mostly thanks to improvements in hardware, such as \acp{GPU}, and the vast amount of data generated every day.
However, large datasets are typically not readily available in clinical settings because of privacy concerns and expensive annotation.
% How we're solving them
In this thesis, we leverage techniques such as transfer learning, self-supervised learning and semi-supervised learning to overcome these low-data regimes.

This thesis is structured in six chapters.
The order of \cref{chap:videos,chap:svt,chap:resection} mimics the temporal order of the associated elements of the clinical pathway.
The literature review related to each topic is presented in the corresponding chapters.

In the \textbf{current chapter}, an overview of epilepsy was first presented.
Then, the relevant elements in the clinical pathway for the treatment of epilepsy and their corresponding challenges were described.

The \textbf{second chapter} (\nameref{chap:videos}),

The \textbf{third chapter} (\nameref{chap:svt}),

The \textbf{fourth chapter} (\nameref{chap:resection}),

The \textbf{fifth chapter} (\nameref{chap:torchio}),

Finally, the \textbf{sixth chapter} (\nameref{chap:conclusion}), concludes the thesis and outlines potential future research directions.
