\subsection{Resective surgery}

If the \ac{iEEG} findings enable a definitive localization of the \ac{EZ}, surgery to resect the determined \ac{EZ} may be performed.
Currently, 40\% to 70\% of patients with refractory focal epilepsy are seizure-free%
\footnote{In this context, ``seizure freedom'' refers to one year without seizures from the surgery.}
after surgery \cite{jobst_resective_2015}.
This is, in part, due to limitations identifying the \ac{EZ}.
Retrospective studies relating presurgical clinical features and resected brain structures to surgical outcome provide useful insight to guide \ac{EZ} resection \cite{jobst_resective_2015}.
To quantify resected structures, the resection cavity, which is mostly composed of \ac{CSF} (\cref{fig:cavities}), must be segmented on the postoperative \ac{MRI}.
A preoperative image with a corresponding brain parcellation can then be registered to the postoperative \ac{MRI} to identify resected structures.

3D manual segmentation of brain resection cavities is a time-consuming process requiring highly trained individuals, and a high inter-rater variability is usual \cite{havaei_brain_2017}.
A tool for automatic segmentation would facilitate and accelerate the research to better understand the relation between the clinical features and surgical outcomes.
% However, automatic techniques for cavity segmentation are challenging as other structures in the brain may look similar, such as the ventricles, cysts or oedemas (\cref{fig:}).
% Moreover, brain shift can happen during surgery, creating parts of the images that are also filled with \ac{CSF}.
Our work on automatic segmentation of brain resection cavities is presented in \cref{chap:resection}.


\newcommand{\plotcavities}[2]{
  \begin{subfigure}{0.8\linewidth}
    \includegraphics[trim=0 0 0 50, clip, width=\linewidth]{percentiles/#1}
    \caption{#2}
  \end{subfigure}
}

\begin{figure}
  \centering

  \plotcavities{p_050_0185_arrows}{Resection cavity in the temporal lobe}
  \plotcavities{p_075_1263_arrows}{Resection cavity in the frontal lobe}
  \plotcavities{p_025_0039_arrows}{Resection cavity in the parietal lobe}

  \caption[Examples of brain resection cavities after curative epilepsy surgery.]{
    Examples of brain resection cavities (marked with green arrows) after curative epilepsy surgery.
  }
  \label{fig:cavities}
\end{figure}
