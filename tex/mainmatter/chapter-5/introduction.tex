\section{Introduction}

Recently, deep learning has become a ubiquitous research approach for solving image understanding and analysis problems.

\Acp{CNN} have become the state of the art for many medical imaging tasks including segmentation~\cite{cicek_3d_2016}, classification~\cite{lu_multimodal_2018}, reconstruction~\cite{chen_variable-density_2018} and registration~\cite{shan_unsupervised_2018}.
Many of the network architectures and techniques have been adopted from computer vision.

Compared to 2D \ac{RGB} images typically used in computer vision, processing of medical images such as \ac{MRI}, \ac{US} or \ac{CT} presents different challenges.
These include a lack of labels for large datasets, high computational costs (as the data is typically volumetric), and the use of metadata to describe the physical size and position of voxels.

Open-source frameworks for training \acp{CNN} with medical images have been built on top of TensorFlow~\cite{abadi_tensorflow_2016,pawlowski_dltk_2017,gibson_niftynet_2018}.
Recently, the popularity of PyTorch~\cite{paszke_pytorch_2019} has increased among researchers due to its improved usability compared to TensorFlow~\cite{he_state_2019}, driving the need for open-source tools compatible with PyTorch.
To reduce duplication of effort among research groups, improve experimental reproducibility and encourage open-science practices, we have developed TorchIO: an open-source Python library for efficient loading, preprocessing, augmentation, and patch-based sampling of medical images designed to be integrated into deep learning workflows.

TorchIO is a compact and modular library that can be seamlessly used alongside higher-level deep learning frameworks for medical imaging, such as the \ac{MONAI} \cite{cardoso_monai_2022}.
It removes the need for researchers to code their own preprocessing pipelines from scratch, which might be error-prone due to the complexity of medical image representations.
Instead, it allows researchers to focus on their experiments, supporting experiment reproducibility and traceability of their work, and standardization of the methods used to process medical images for deep learning.
